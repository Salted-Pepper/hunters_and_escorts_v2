\documentclass[12pt]{article}
% Double spacing
\usepackage{setspace}
\doublespacing
% Font encoding and input
\usepackage[utf8]{inputenc}
% Table and text packages (from main.tex)
\usepackage{longtable}
\usepackage{blindtext,alltt}
% Title and author
\title{Hide and Seek in the Philippine Sea: China's Ability to Detect U.S. Surface Ships using Drones and Irregular Forces}
\author{Matthew Cancian \\ Rob Kieneker}
\date{}

\begin{document}
\maketitle

\begin{abstract}
How likely are U.S. Navy ships to be found by Chinese drones and irregular ships during a war? If U.S. Navy ships cannot hide, it's unlikely that they will be make operationally significant contributions during a war. While satellites and over-the-horizon backscatter radars are the central tools in China's ISR toolkit, drones and irregular ships are also mentioned but never analyzed. This paper builds an agent based model of these assets and simulates their effectiveness, factoring in the Chinese order of battle, maintenance and travel time, plausible search patterns, and realistic sensor effectiveness. We find that U.S. ships have a 50\% chance of being detected within 24 hours of getting within 2,000 nautical miles of China. The implication of this is that, even should the United States negate Chinese satellites and ground-based radars, surface ships will be unlikely to conduct missions without taking fire from China. Methodologically, this paper contributes to the state of the art in modeling operational problems.
\end{abstract}

% Outline with topic sentences as intro paragraphs
\section{Introduction}
How likely are U.S. Navy ships to be found by Chinese drones and irregular ships during a war?

If U.S. Navy ships cannot hide, it's unlikely that they will be make operationally significant contributions during a war.

While satellites and over-the-horizon backscatter radars are the central tools in China's ISR toolkit, drones and irregular ships are also mentioned but never analyzed.

This paper builds an agent based model of these assets and simulates their effectiveness, factoring in the Chinese order of battle, maintenance and travel time, plausible search patterns, and realistic sensor effectiveness.

We find that U.S. ships have a 50\% chance of being detected within 24 hours of getting within 2,000 nautical miles of China.

The implication of this is that, even should the United States negate Chinese satellites and ground-based radars, surface ships will be unlikely to conduct missions without taking fire from China.

Methodologically, this paper contributes to the state of the art in modeling operational problems.

\noindent How likely are U.S. Navy ships to be found by Chinese drones and irregular ships during a war?

\noindent If U.S. Navy ships cannot hide, it's unlikely that they will be make operationally significant contributions during a war.

\noindent While satellites and over-the-horizon backscatter radars are the central tools in China's ISR toolkit, drones and irregular ships are also mentioned but never analyzed.

\noindent This paper builds an agent based model of these assets and simulates their effectiveness, factoring in the Chinese order of battle, maintenance and travel time, plausible search patterns, and realistic sensor effectiveness.

\noindent We find that U.S. ships have a 50\% chance of being detected within 24 hours of getting within 2,000 nautical miles of China.

\noindent The implication of this is that, even should the United States negate Chinese satellites and ground-based radars, surface ships will be unlikely to conduct missions without taking fire from China.

\noindent Methodologically, this paper contributes to the state of the art in modeling operational problems.

\section{Literature Review}

\section{Methodology}

\section{Results}

\section{Analysis}

\section{Conclusion}

\end{document}
